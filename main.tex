%%%%%%%%%%%%%%%%%%%%%%%%%%%%%%%%%%%%%%%%%
% Friggeri Resume/CV
% XeLaTeX Template
% Version 1.2 (3/5/15)
%
% This template has been downloaded from:
% http://www.LaTeXTemplates.com
%
% Original author:
% Adrien Friggeri (adrien@friggeri.net)
% https://github.com/afriggeri/CV
%
% License:
% CC BY-NC-SA 3.0 (http://creativecommons.org/licenses/by-nc-sa/3.0/)
%
% Important notes:
% This template needs to be compiled with XeLaTeX and the bibliography, if used,
% needs to be compiled with biber rather than bibtex.
%
%%%%%%%%%%%%%%%%%%%%%%%%%%%%%%%%%%%%%%%%%

\documentclass[]{cv-style} % Add 'print' as an option into the square bracket to remove colors from this template for printing

%\addbibresource{bibliography.bib} % Specify the bibliography file to include publications

\sethyphenation[variant=british]{english}{} 

%\usepackage{titlesec}
%\titlespacing{\section}{0pt}{\parskip}{-\parskip}

\begin{document}

\header{stefan}{wojcik} % Your name and current job title/field

%----------------------------------------------------------------------------------------
%	SIDEBAR SECTION
%----------------------------------------------------------------------------------------

\begin{aside} % In the aside, each new line forces a line break
\section{contact}
~
\href{mailto:stefan.j.wojcik@gmail.com}{stefan.j.wojcik@gmail}
%\href{http://spot.colorado.edu/~stwo0664/}{http://spot.colorado.edu/~stwo0664/}
\href{http://twitter.com/stefanjwojcik}{tw://stefanjwojcik}
\section{programming languages}
{\color{red} $\varheartsuit$} Julia,
{\color{red} $\varheartsuit$} Python,
{\color{red} $\varheartsuit$} R,
SQL
 \& Bash
\section{natural languages}
english mother tongue
spanish - fluent
portuguese - fluent
\end{aside}


\section{about}

I'm a Computational Social Scientist specializing in using large-scale digital trace data to understand human behavior. 

%----------------------------------------------------------------------------------------
%	EDUCATION SECTION
%----------------------------------------------------------------------------------------

\section{education}

\begin{entrylist}

%------------------------------------------------

\entry
{2010--2014}
{PhD {\normalfont of Political Science}}
{The University of Colorado, Boulder}
{Comparative Politics, Statistical Methods}
%\\ My thesis used computational methods to understand political networks in Brazil.

%------------------------------------------------

\entry
{2008--2010}
{Master {\normalfont of Political Science}}
{The University of Wisconsin, Milwaukee}
{Comparative Politics}

%------------------------------------------------

\entry
{2003--2007}
{Bachelor {\normalfont of Political Science, Spanish}}
{The University of Minnesota, Duluth}
{Specialized in Political Theory}

%------------------------------------------------

\end{entrylist}

%----------------------------------------------------------------------------------------
%	WORK EXPERIENCE SECTION
%----------------------------------------------------------------------------------------

\section{experience}

\begin{entrylist}

\entry
{2025--NOW}
{NANOCENTURY AI}
{Eau Claire, WI}
{ \jobtitle{Founder/Principal Data Science Consultant} \\
Conceived and executed custom research and analysis for top industry and academic clients, including R1 research institutions and a global 500 social media company. Built bleeding edge AI-ready research deliverables. 
%
}
   

\entry
{2023--2025}
{NATIONAL GRID RENEWABLES/(GERONIMO)}
{Remote}
{ \jobtitle{Principal Data Scientist} \\
Led a small team to develop an ML system for energy price forecasting and automated battery trading, increasing annual revenue capture for renewable assets by 20–30\%.
  
%
}

%------------------------------------------------

\entry
{2019--2023}
{TWITTER}
{Various Locations}
{ \jobtitle{Staff Quant/Data Scientist} \\
Led industry-defining research to measure and improve the health of the Twitter platform. I led quant research on the 2020 US Elections, generative AI interventions, and was a lead author of the Community Notes/Birdwatch research, which became industry standard reading.
}


\entry
{2017--2019}
{THE PEW RESEARCH CENTER}
{Washington, DC}
{ \jobtitle{Computational Social Scientist} \\
I conducted surveys of online populations, and applied a variety of modern machine learning and statistical tools to digital trace data, including text, images, and survey data. Utilized deep learning models, support vector machines, hierarchical models, and tree-based models. \\ \emph{-Media trained}} \\

%------------------------------------------------

\entry
{2014--2016}
{THE LAZER LAB}
{Harvard University/ Northeastern University}
{\emph{Postdoctoral Fellow of Computational Social Science} \\
Forecasted global elections, tracked flu rates in collaboration with Microsoft Research, developed methods to extract representative data from Twitter to predict public opinion, flu, movie revenues.}

\end{entrylist}


%----------------------------------------------------------------------------------------
%	PUBLICATIONS SECTION
%----------------------------------------------------------------------------------------

\section{selected publications}

\begin{itemize}
\item 2024 ``Facing Voters: Gender Expression, Gender Stereotypes, and Vote Choice''. \textbf{Journal of Politics}. (with Shawnna Mullenax). \href{https://www.journals.uchicago.edu/doi/full/10.1086/729958}{LINK} 

\item 2022 ``Birdwatch: Crowd Wisdom and Bridging Algorithms can Inform Understanding and Reduce the Spread of Misinformation''. \textbf{Arxiv}. (with Sophie Hilgard, Nick Judd, Delia Mocanu, Stephen Ragain, M.B. Fallin Hunzaker, Keith Coleman, and Jay Baxter). \href{https://arxiv.org/abs/2210.15723}{LINK} 

\item 2021 ``Survey data and human computation for improved flu tracking''. \textbf{Nature Communications}. (with Avleen Bijral, et al.). \href{https://www.nature.com/articles/s41467-020-20206-z}{LINK}. 

\item 2019 ``The challenges of using machine learning to identify gender in images''. \textbf{Pew Research Center}. (with  Emma Remy). \href{https://www.pewresearch.org/internet/2019/09/05/the-challenges-of-using-machine-learning-to-identify-gender-in-images/}{LINK}. 

\item 2019 ``Sizing Up Twitter Users''. \textbf{Pew Research Center}. (with  Adam Hughes). \href{https://www.pewinternet.org/2019/04/24/sizing-up-twitter-users/}{LINK}. 

%\item 2019 ``Gender and Jobs in Online Image Searches''. \emph{Pew Research Center}. (with Onyi Lam, Adam Hughes, and Brian Broderick). \href{https://www.pewsocialtrends.org/2018/12/17/gender-and-jobs-in-online-image-searches/}{LINK}. 

\item 2018 ``Bots in the Twittersphere''. \textbf{Pew Research Center}. (with  Solomon Messing, Aaron Smith, Lee Rainie and Paul Hitlin). \href{http://www.pewinternet.org/2018/04/09/bots-in-the-twittersphere/}{LINK}. 

\item 2017 ``Do Birds of a Feather Vote Together, or Is It Peer Influence?".  \textbf{Political Research Quarterly}. 

\item 2017 ``Are Democracies Cheaper? Regime Type, Electoral System, and Consumer Price Levels" (with Andy Baker).  \textbf{International Political Science Review}.

\item 2017 ``Voters of the Year: 19 Voters Who Were Unintentional Election Poll Sensors on Twitter " (with Will Hobbes, Lisa Friedland, Kenneth Joseph, Oren Tsur, and David Lazer).  \textbf{ICWSM} 2017.

\item 2017 ``Improving Election Prediction Internationally". (with Ryan Kennedy and David Lazer). \textbf{Science} Vol. 355 Issue 6324: 515-520. 

\item 2017 ``Men Idle, Women Network: How Networks Help Female Legislators Succeed". (with Shawnna Mullenax). Winner of Best Graduate Student Paper, Department of Political Science, CU-Boulder. \textbf{Legislative Studies Quarterly}.

\item 2016 ``Political Networks and Computational Social Science``. (with David Lazer).  \textbf{Oxford Handbook of Political Networks}. 

%\item 2016 ``Gaming for science: A demo of online experiments on VolunteerScience. com'' \emph{Proceedings of the 19th ACM Conference on Computer Supported Cooperative Work and Social Computing Companion. pp. 86-89.}

%\item 2016 ``Why Legislative Networks? Analyzing Legislative Network Formation``.  \textbf{Political Science Research and Methods}. 

%\item 2015 ``The Changing Role of Ministers in the Legislative Agenda in Brazil''. (with Lucio Renn\'o). RIEL - Revista Ibero-Americana de Estudos Legislativos. Number 4, May 2015. Rio De Janeiro, FGV. 

\end{itemize}

%----------------------------------------------------------------------------------------
%	COMMUNICATION SKILLS SECTION
%----------------------------------------------------------------------------------------

\section{selected presentations}

\begin{entrylist}

%------------------------------------------------

\entry 
{2024}
{Invited Talk}
{Google IQ Talks}
{Birdwatch:Crowd Wisdom and Bridging Algorithms can Inform Understanding and Reduce the Spread of Misinformation}

\entry 
{2024}
{Invited Talk}
{University of Colorado Denver Symposium}
{Birdwatch:Crowd Wisdom and Bridging Algorithms can Inform Understanding and Reduce the Spread of Misinformation}

\entry
{2020}
{Conference Presentation}
{International Conference on Computational Social Science}
{Mind Map: A Bootstrapping Approach to Topic Extraction}

\entry
{2019}
{Methods Presentation}
{The World Bank}
{Co-presented research conducted about gender representation in online news.}

%------------------------------------------------

%\entry
%{2018}
%{Oral Presentation}
%{European External Action Service}
%{Presented research conducted about bots on Twitter.}

%------------------------------------------------

\end{entrylist}

%----------------------------------------------------------------------------------------
%	AWARDS SECTION
%----------------------------------------------------------------------------------------

\section{selected awards}
\begin{itemize}
\item National Science Foundation Political Networks Conference Fellowship, Portland, Oregon, June 2015, total \$700
%\item Best Graduate Student Paper (with Shawnna Mullenax), University of Colorado, 2014, total \$100
\item Graduate School Summer Fellowship, University of Colorado, 2014, total \$6000
\item Beverly Sears Research Award, University of Colorado, 2014, total \$1000
%\item Latin American Research Cluster Grant, University of Colorado 2013, 2014, total \$250
\item Political Science Summer Research Grant, University of Colorado 2012, 2013, 2014, total \$1800, \$2500, \$2250
\end{itemize}


%\printbibsection{article}{article in peer-reviewed journal} % Print all articles from the bibliography
%\printbibsection{book}{books} % Print all books from the bibliography
%\begin{refsection} % This is a custom heading for those references marked as "inproceedings" but not containing "keyword=france"
%\nocite{*}
%\printbibliography[sorting=chronological, type=inproceedings, title={international peer-reviewed conferences/proceedings}, notkeyword={france}, heading=bibheading]
%\end{refsection}
%\begin{refsection} % This is a custom heading for those references marked as "inproceedings" and containing "keyword=france"
%\nocite{*}
%\printbibliography[sorting=chronological, type=inproceedings, title={local peer-reviewed conferences/proceedings}, keyword={france}, heading=bibheading]
%\end{refsection}
%\printbibsection{misc}{other publications} % Print all miscellaneous entries from the bibliography
%\printbibsection{report}{research reports} % Print all research reports from the bibliography
%----------------------------------------------------------------------------------------

%----------------------------------------------------------------------------------------
%	INTERESTS SECTION
%----------------------------------------------------------------------------------------

%\section{interests}

%\textbf{professional:} data analysis, company profiling, risk analysis, economics, web design, web app creation, software design, marketing \textbf{personal:} piano, chess, cooking, dancing, running

\end{document}